% 1 de abril
Lie Derivatives were defined by
\begin{enumerate}
\item $\displaystyle L_xf = X(f) = \lim_{t\to 0} \left[ \frac{\Phi_t^* f-f}{t} \right]_p \quad (\text{in } \RRR)$
\item $\displaystyle (L_x\omega)_p =  \lim_{t\to 0} \left[ \frac{\Phi_t^* \omega-\omega}{t} \right]_p $ (in $\tnsp^*_p M$)
\item $\displaystyle (L_xY)_p = \lim_{t\to 0} \left[ \frac{\Phi_t^* (Y_{\Phi(p)})-Y_p}{t} \right]_p $in $\tnsp_pM$
\end{enumerate}

\begin{ddef}[Leibnitz Rule]
The Leibnitz rule is:
$$L_X[\omega(Y)] = (L_x\omega) Y + \omega(L_xY)$$
\end{ddef}


\begin{lema}
Let $\Phi \from (-\epsilon, \epsilon) \times U \to M$ be a the local flow of $X\in X^\infty(M)$ near $p\in M$. Given $f\in C^\infty(M)$, let:
$$\eta: (-\epsilon,\epsilon)\times U \to \RRR$$
defined by:
$$\eta_t(q) = \eta(t,q) = \int_0^1 [D_1(f\circ \Phi)] (-st,q) ds$$
Then $f\circ \Phi_{-t} = f-tq$ and $\eta_a= Xf$.
\begin{proof}
    \begin{enumerate}
        \item $t\eta_t(q) = \int_0^1 t[D_t(f\circ \Phi)(-st,q) ds$
                $$= \int_0^1 -\frac{\partial }{\partial s} (f\circ \Phi (-st,q))ds$$
            \item $\displaystyle Xf(a) = \frac{d}{dt} f\circ \phi_t(a) |_{t=0} = -\frac{d}{dt} f\circ \Phi_{-t} (a) |_{t=0}$
                $$\frac{d}{dt}\left( t\eta_t(a) - f(a)\right) |_{t=0}$$
        \end{enumerate}
        \underline{Proof of the Proposition}
        For $f\in C^\infty (M)$ we have:
        \begin{gather*}
            d\Phi_{-t} (Y_{\phi_t(p)}) = Y(f\circ \Phi_{-t}) \circ \Phi_t(p)\\
            = (Yf - tY\eta_t)\circ \Phi_t(p)
        \end{gather*}
        So 
    \begin{gather*}
    (L_XY)_p f = \lim_{t\to 0} \frac{d\Phi_{-t} (Y_{\Phi_t(p)}f - Y_p f)}{t}\\
    = \lim_{t\to 0} \left[ \frac{(Yf) \circ \Phi_t(p) - (Yf)_p}{t} - (Y\eta_t)\circ \Phi_t(p) \right]\\
    = X_p(Yf) - Y_p(Xf) = [X,Y]_p f
\end{gather*}
\end{proof}
\end{lema}
 
\begin{teorema}
$$L_XY = [X,Y]$$
\end{teorema}

\begin{teorema} 
    Let $X,Y \in X^\infty (M)$ induce that local flows $\Phi_t,\Psi_s$. Then $X,Y$ commute near $p$ iff $\Phi_t$ and $\Psi_s$ also do.\\
    i.e. 
    \begin{gather}
        [X,Y] = 0 \text{ near } p \label{equ:1}\\
        \iff \Phi_t\circ \Psi_s = \Psi_s \circ \Phi_t  \text{ near } p \text{ for small } s,t \label{equ:2}
    \end{gather}
    \begin{proof}
        $\ref{equ:2} \implies \ref{equ:1}$\\
        For $q$ near $p$ \ref{equ:2} implies:
        \begin{gather*}
        \frac{\partial}{\partial s} \Phi_t\circ \Psi_s(q) | _{s=0} = \frac{\partial }{\partial s} \Psi_s \circ \Phi_t(q) |_{s=0}\\
        d\Phi_t(Y_q) = Y_{\Phi_t(q)} \implies Y_q = d\Phi_{-t} (Y_{\Phi_t(q)}) \implies (L_x Y)_q \\
        = \lim_{t\to 0} \frac{Y_q - Y_q}{t} =0
    \end{gather*}
    $\ref{equ:1} \implies \ref{equ:2}$ this direction is harder.\\
    For $q$ near $p$, let $\gamma(t)$ be the curve in $\tnsp_qM$ then $\gamma(t) = d\Phi_{-t} (Y_{\Phi_t(q)})$, for small $t$. Then 
    \begin{gather*}
    \gamma'(t) = \lim_{h \to 0} \left[ \frac{\gamma(t+h) - \gamma(t) }{h} \right]\\
    = \lim_{h \to 0}\frac{1}{h} \left[ d\Phi_{-t-h} (Y_{\Phi_{t+h}(q)}) -  d\Phi_{-t} (Y_{\Phi(q)})\right]\\
    d\Phi_{-t} \left(\lim_{h\to 0}\frac{1}{h} \left[  d\Phi_{-h}(Y_{\Phi_h(\Phi_t(q))}) -  Y_{\Phi(q)}\right] \right)\\
\end{gather*}
This is just the Lie derivative
$$ d\Phi_{-t} \left((L_x Y)_{\Phi_t(q)}\right) = d\Phi_{-t}\left([X,Y]_{\Phi_t(q)}\right) =0$$
and we know that $[X,Y]=0$. So, $\gamma'(t) \equiv 0$ and $\gamma(t) = \gamma(0) = Y_q$. This is only a point it doesn't go anywhere; i.e. 
$$d\Phi_{-t} \left(Y_{\Phi_t(\delta(s))}\right) = Y_{\delta(s)}$$
    for any curve $\delta(s)$ and $q$ near $p$.\\
    In particular, consider the curve $\gamma(s) \subset M$:
    $$\delta(s) = \Phi_{-t} \circ \Phi_s \circ \Phi_t(q) $$
    Then $\delta(0) = q$, and 
    \begin{gather*}
    \delta'(s) = \frac{d}{dh}\delta(s+h) |_{h=0} =  \frac{d}{dh}\Phi_t \circ \Psi_{h+s} \circ \Phi_t (q) |_{h=0}\\
    =  \frac{d}{dh} \Phi_{-t} \circ \Psi_h \circ \Phi_t \circ \left( \Phi_{-t} \circ \Psi_s \circ \Phi_t(q)\right)|_{h=0}\\
=  \frac{d}{dh} \Phi_{-t} \circ \Psi_h \circ \Phi_t (\delta(s))|_{h=0}   =  \frac{d}{dh} \Phi_{-t}\left( \Psi_h\left( \Phi_t \left(\delta(s) \right)\right)\right)\\
= d\Phi_{-t} (Y_{\Phi_t(\delta(s))}) = Y_{\delta(s)} = \delta'(s)
\end{gather*}
$\implies \delta(s)$  is the integral curve of $Y$ through $q$.
    \end{proof}
\end{teorema}

\begin{teorema}
    Let $X\in X^\infty(M)$ and $X_p\neq 0 $, for some $p\in M$ then there exists a chart 
    $$\phi=(x_1,x_2,\ldots,x_n)\from U\to \RRR^n$$
    $p\in U$, such that $X$ is $\phi-$related to $\frac{\partial}{\partial r_1}$ i.e. $X=\frac{\partial}{\partial X_1}$
\end{teorema}
