% 17 de abril
``Lie groups are orientable''\\
Say two ordered bases of vector spaces have the same orientable if the Change of Basis matrix has positive determinate.

Define an orientation of vector spaces to be an equivalence class of this relationship.

\begin{ddef}
A \textbf{Frame} on an open set $U\subset M^n$ is an $n$-tuple $(x_1,\ldots, x_n)$ is (possibly disjoint) vector fields on $U$ such that $\forall \, p\in U\; (x_1(p),\ldots, x_n(p))$ is an ordered basis for $\tnsp_p M$.
\end{ddef}

\begin{itemize}
\item Say a manifold is paralelizable if there exists a smooth global frame i.e. if there are $x_1,\ldots, x_n$ smooth vectors fields such that $\forall\, p\in M([X_1(p), \ldots, X_n(p)]) $ is a basis for $\tnsp_p M$.
\item Let $(V_1, \ldots, V_n)$ be a basis for $\tnsp_e 0$ define $V_j$ by 
$$V_j (g) = L_{g^*} (V_j)$$
\end{itemize}

DoCarmo problem \#1\\
The antipodal mapping $A\from S^n \to S^n$ $ p\mapsto -p$ is an isometry.\\
To show that $A$ is an isometry: $dA = -Id$
\begin{gather*}
\Gamma \from S^n \to \RRR P^n\\
p\mapsto [p]\\
p\in S^n , \quad u,v \in \tnsp_p S^n
\end{gather*}
We define: $\langle d\pi_p(u),d\pi_p(v) \rangle_{\pi(p)} = \langle u | v\rangle_p$ checking if the metric is well-defined. If $\pi = \pi\circ A$ this implies that
$$d\pi_{-p} = d\pi_p \circ dA_{A(p)}$$
then
$$\langle d\pi_{-p}(-u), d\pi_{-p}(-v)\rangle = \langle d\pi_{-p}\circ dA(u) | d\pi_{-p} \circ dA(v) \rangle$$

\subsection{Matrix Lie Groups}
$GL_n(\RRR) \subset \RRR^{n^2}$ Left invariant vector field.
$$A\in \lie{g} = gl_n(\RRR) \to X_m= M\,A$$
We define the commutator operator 
\begin{gather*}
[A,B] = A\,B - B\,A
\Ad_M U= M UM^{-1}
\end{gather*}

if $e^{tX} = M(t)$ and $M(0)=Id$ and $M'(0)=X\in \lie{g}$
\begin{gather*}
    \left.\frac{d}{dt} \Ad_{M(t)} U \right|_{t=0} =     \left.\frac{d}{dt}\left( MUM^{-1} \right) \right|_{t=0}\\
    = \dot M U M^{-1} + M U (\dot M^{-1})|_{t=0} = \dot M_0 U - U\dot M_0 = XU-UX= [X,U] = \ad_X(U)\\
    \dot M M^{-1} = 0 \\
    (\dot M ^{-1}) = - M^{-1}\dot M M^{-1}
\end{gather*}
assuming Ad-invariance
$$ \langle \Ad_M U | \Ad_M V \rangle = \langle U, V \rangle \implies \langle [X,U] | V \rangle + \langle U|[X,V] \rangle$$

\subsection{Covariant Derivative}
\begin{ddef}
    Given smooth vector fields $X,Y\in X^\infty(M)$, we define the \emph{covariant derivative} (affine connection) a new vector field $\nabla_xY \in X^\infty(M) $ that satisfies:
    \begin{enumerate}
        \item $\nabla_x(Y+Z) = \nabla_X Y + \nabla_XZ$
        \item $\nabla_{X+Y} Z = \nabla_X Z + \nabla_Y Z $
        \item $\nabla_{fX}Y = f\nabla_XY,\; f\in C^\infty(M)$
        \item $\nabla_X fY = X(f) Y + f\nabla_XY$
    \end{enumerate}
\end{ddef}
    Comparing $L_XY $ and  $\nabla_XY$ write $\phi= (X_1,X_2,\ldots, X_n)$ 
    \begin{gather*}
        Y=  \sum b^j \frac{\partial}{\partial x_j} \\
        \nabla_XY = \nabla_X \sum_j b^j \frac{\partial}{\partial x_j}\\
        = \sum_j X(b^j) \frac{\partial}{\partial x_j} + b^j\nabla_X \left.  \frac{\partial}{\partial x_j} \right|_p
    \end{gather*}
    Where $X= \sum a^i  \frac{\partial}{\partial x_i}, \ Y = \sum b^j \frac{\partial}{\partial x_j}$.

    \begin{ddef}[Cristoffel Symbol]
        \begin{gather*}
            \Gamma_{i\,j}^k: \nabla_{ \frac{\partial}{\partial x_i}} \frac{\partial}{\partial x_j} = \sum_k \Gamma_{i\,j}^k  \frac{\partial}{\partial x_k}\\
            \nabla_X Y = \sum_{i\, j} a^i\left(  \frac{\partial b^j}{\partial x^i}  \frac{\partial}{\partial x_i} + \sum_k b^j \Gamma_{ij}^k  \frac{\partial}{\partial x_k}\right)
        \end{gather*}
    \end{ddef}
    $Y= Y(t)$ is a smooth vector fields alogn the curve $\gamma(t)$ is parallel along $\gamma$ if:
    \begin{gather*}
        \frac{D}{dt}Y = \nabla_{\dot \gamma}Y =0\\
        \frac{D}{dt}Y = \sum_k \left( b^k + \sum_{ij} b^j\gamma^i \Gamma_{ij}^k \right) \frac{\partial}{\partial x^k} 
    \end{gather*}
    For $Y$ parallel: $0=\dot b^k + \sum_{ij} b^j \dot \gamma^k  \Gamma_{ij}^k$ for $k=1,\ldots, n$.
    \begin{ddef}
        The smooth curve $\gamma(t)$ is a geodesic iif:
        \begin{gather*}
            \frac{D}{dt} \dot \gamma = \nabla_{\dot \gamma} \dot \gamma =0\\
            b^k \longrightarrow \dot \gamma^k = (\dot{x_k\circ \gamma})
        \end{gather*}
    \end{ddef}
