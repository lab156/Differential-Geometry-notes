%15 de abril
New scientist wrinkled doughnut solves geometrical mystery.

\begin{teorema}[Gromov]
There exists a $c^1-$ison embedding of the flat torus in $\RRR^3$.
\end{teorema}

Continuing with the last lecture, we have:
\begin{gather*}
\Ad_g X = dR_{g-1}dL_g(X)\\
\ad_XY = \frac{d}{dt} \Ad_{e^{tX}}Y|_{t=0} \overset{\text{Prop}}{=} L_XY \overset{\text{Prop}}{=}[X,Y]
\end{gather*}

\begin{ddef}
A \textbf{Riemannian metric} $\langle,\rangle$ on $G$ is left invariant if, for any $g\in G$ $L_g\from G\to G$ is an isometry i.e. $L^{-1}_g\langle\ \rangle = \langle\ \rangle$, i.e.
\begin{gather}
\forall g,h\in G,X,Y\in T_h G\label{eqs}\\
\langle X,Y\rangle = \langle dL_gX, dL_gY\rangle \label{eqss}
\end{gather}
\end{ddef}
Specialized (\ref{eqs}) to $h=f,\  g=f^{-1}, \ X=U,\ Y=V$
$$\langle U, V \rangle_f  \overset{\text{Using (\ref{eqss})}}{=} \langle dL_{f^{-1}}(U)|  dL_{f^{-1}}(V) \rangle_e$$

To prove the first direction ((\ref{eqs}) $\implies$ (\ref{eqss})). Try to define left-invariant metric from a given one at $e$: Assume $\langle \; \rangle_e$ use (\ref{eqss}) to define $\langle , \rangle_f$. 

On the other hand if we assume (\ref{eqss}) and want to prove (\ref{eqs}).
\begin{enumerate}[1)]
    \item $f=h,\ U=X,\ V = Y$: 
        $$ \langle X, Y \rangle_L = \langle d\, L_{h^{-1}} X, d\, L_{h^{-1}}Y\rangle_e$$
    \item $f=gh,\ U = d\, L_gX,\ V=d\, L_g Y,\ X,Y\in \tnsp_hG$
        \begin{align*}
            \langle d\, L_gX, d\,L_gY\rangle_{gh} &= \langle d\, L_{f^{-1}} U , d\, L_{f^{-1}} V \rangle_e \\
                                                  &= \langle d\,L_{h^{-1}g^{-1}} d\,L_g X , d\, L_{h^{-1}g^{-1}} dL_g Y \rangle_e\\
                                                  &= \langle d\, L_{h^{-1}}X, d\, L_{h^{-1}} Y \rangle_e 
        \end{align*}
        And this implies the LHS (\ref{eqs}) = RHS (\ref{eqs})
\end{enumerate}
Likewise a metrix is right-invariant if for each $g\in G$ and $R_g: G\to G$ is an isometry. We use $\forall g,h \in G$ where $X,Y\in \tnsp_h G$. 
\begin{gather*}
    \langle X|Y \rangle_h \overset{(\ref{eqss})}{=} \langle d\, R _{g^{-1}} X | d\, R_gY \rangle_{g^{-1}h} \\
    \langle X|Y\rangle_h \overset{(\ref{eqs})}{=} \langle d\, L_g X | d\, L_g Y \rangle _{gh}
\end{gather*}
Now suppose that $\langle\ ,\ \rangle$ is a bi-invariant (left and right invariant) then $\langle \ \rangle_e$ is  $\Ad$-invariant.

\begin{gather*}
\forall g\in G,\ X,Y\in \lie{g}\simeq \tnsp_e G  \\
\langle \Ad_g X | \Ad_g Y \rangle_e = \langle d\,R_{g^{-1}}dL_g X | d\, R_{g^{-1}} d\, L_g Y \rangle_e\\
\overset{h=g}{=} \langle dL_g X | dL_g Y \rangle_g \overset{h=e}{=} \langle X| Y\rangle_e
\end{gather*}
conversely, if a metric on $\tnsp_eG = \lie{g}$ is Ad-invariant and we define $\langle\, \rangle$ on $G$. To be left-invariant, then the result is also right invariant.

\begin{teorema}
    If $\langle \ , \  \rangle$ is a bi-invariant metric on $G$ and $V\in g = \tnsp_e G$; then $\ad_V$ is skew-adjoint with resp to $\langle \, \rangle$ i.e. $\forall\, U,W\in \lie{g}$
    $$\langle \ad_V U, W\rangle \underset{\text{sk. adj.}}{=} - \langle U, \ad_V W \rangle$$
DoCarmo writes that for $e$:
$$\langle [U,V],W \rangle = \langle U, [V,W]\rangle \overset{h=e}{=} \langle X|Y \rangle_e$$
Compare to the scalar triple product in $\RRR^3$
$$U\times V\cdot W = U\cdot V \times W$$
\begin{proof}
    Let $g= e^{tV}$, then 
    \begin{gather*}
        0=\left. \frac{d}{dt}\langle U, V \rangle \right |_{t=0}\\
        = \left. \frac{d}{dt} \langle \Ad_gU, \Ad_g W \rangle \right|_{t=0}\\
    \end{gather*}
    Conversely, condition skew-adjoint can be shown to imply Ad-invariance. Hence, a metric on satisfying Self-Adjoint induces a bi-invariant metric on $G$.
\end{proof}
\end{teorema}

\begin{ddef}
    The \emph{Cartan-Killing} form $K\from \lie{g}\times \lie{g} \to \RRR$ is the symmetric bilinear form with formula:
    $$K(X,Y) = \trace(\ad_X \ad_Y)$$
    \end{ddef}
    \begin{teorema}
        $K$ is self-adjoint i.e. 
        $$K([U,V],W)=K(U,[V,W])$$
        \begin{proof}
            First note the Jacobi Identity:
            \begin{gather*}
                = [[U,V],W] + [[V,W],U] + [[W,U],V]\\
                \implies \ad_{[U,V]}W = \ad_U \ad_V W - \ad_V\ad_U W
            \end{gather*}
Next note that for linear transformations $A,B,C\from \lie{g} \to \lie{g}$
\begin{gather*}
    \trace(A\, B\, C) = \trace(A^{-1}A\, B\, C\, A) = \trace(B\,C\,A)\\
    K([U,V],W) = \trace(\ad_{[U,V]} \ad_W)\\
        =\trace(\ad_U \ad_V \ad_W - \ad_V\ad_U\ad_W)\\
        =\trace(\ad_U(\ad_V \ad_W - \ad_W \ad_V))\\
        =\trace(\ad_U\ad_{[V,W]}) = K(U,[V,W])
    \end{gather*}
\end{proof}
\end{teorema}

\begin{teorema}
    $K$ satisfies the Self-Adjoint condition $K([U,V],W)= K(U,[V,W])$
    \begin{itemize}
        \item For semi-simple Lie algebras $K$ is non-degenerate.
        \item For compact Lie groups $G,\ K$ is negative definite. So, $\langle,\rangle=-k$ is positive definite.
    \end{itemize}
\end{teorema}

\begin{examples}
    For $G=SO(3)$, get bi-invariant metric
    $$\langle \ | \ \rangle = -k \quad [\ , \ ] \leftrightarrow x$$
    \emph{Great Exercise}: this has to do a lot with rigid body Mechanics
\end{examples}
