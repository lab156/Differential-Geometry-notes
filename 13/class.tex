%27 de Febrero
Last time inmersion embedding immersion doesn't lose any dimension.
\begin{ddef}
If $(\phi,V)$ and $\phi=(x_1,x_2,\ldots, x_m)$ is a chart on $M^n$ a $k-$slice of $\phi$ is a map $(\phi_k,V_k)$, where
\begin{gather*}
V_k = \{p\in V\from x_{k+1} = x_m =0\},\\
\phi_k=(x_1,\ldots, x_k),\quad \phi_k\from V_k \to \RRR^k
\end{gather*}
\end{ddef}

\begin{ddef}
$k\subset M$ is a (Regular) Submanifold of $M$ if $k$ a topological subspaces of $M$ and a smooth manifold with atlas consisting of slices of charts in $\hat \atlas(M)$.\\
in this case $i: k\to M$ is an embedding submanifolds in a weaker sense.
\end{ddef}

\begin{teorema}
Let $M^m$ and $N^n$ be a smooth manifolds (assume $m$ is 2nd countable).\\
Let $f\from M\to N$ be a smooth map; and  let $q\in \NN$ lie in the image if $f$. If $df_p \from \tnsp_pM \to \tnsp_qN$ is sujective at each $p\in f ^{-1} (a)$, then $k= f ^{-1} (a)$ is a (regular) submanifold of $M$ of dimension $k=m-n$ ($q$ is a ``regular-value'').
\end{teorema}

\begin{remarks}
This follows the ``Nullity + Rank'' theorem of linear algebra.\\
If $T\from \RRR^m \to \RRR^n$ linear map then:
$$m=\dim(\ker\,T) + \rank(T)$$
If $T$ is onto then $\rank(T)=n$
$$m=\dim(T ^{-1} (0)) + n$$
so $k=\dim(T ^{-1} (0)) = m-n$
$$k=T ^{-1} (0) = \ker(T)$$
Now if $p\in K$, let 
$$T=df_p \from \tnsp_pM^n \to \tnsp_qN^n$$
we can make the identification:
$$\tnsp_pK \cong \ker(T) \subset \tnsp_pM$$
so $\dim(k)= \dim(\tnsp_pM) = m-n$. Note that Lie groups are submanifolds of Euclidean spaces.
\end{remarks} 

\begin{examples}
Uses of the implicit function theorem $f\from \RRR^{n+1} \to \RRR$, 
$$f(x)=f(x_1,\ldots, x_{n+1})=x\cdot x \sum_{i=1}^{n+1} x_i^2$$
$S^n=f ^{-1} (1)$ note $df_xv = \frac{d}{dt}(x(t)\cdot x(t))|_{t=t_0} = 2x(t)\cdot x(t)|_{t=t_0}$ (where $v=\dot x(t)$) $=2v\cdot x=2x\cdot v$.\\
So $df_xv=2x\cdot v$ is surjective for $x\neq 0$ (just take $v=x$).\\
Then $S^n= f ^{-1} (1)$ is a regular submanifold of $\RRR^{n+1}$ of dimension $(n+1) - 1=n$.\\
Also, $$\tnsp_xS^n= \ker (df_x)=x^\perp=\{v\in \RRR^{n+1} \from x\cdot v=0 \}$$ There is a neat way of embedding (Trick) of $\tnsp S^n$ into $\RRR^{2n+2}$
Take:
$$(x,v) \in \RRR^{2n+2} \mapsto \tnsp S^n \simeq \{(x,v) \in \RRR^{2n+2}\from f(x)=1, \, df_xv=0 \}$$
$=F ^{-1} (0)$, where $F(x,v)=(f(x)-1, df_xv) = (\tilde f(x), f'(x,v))$.\\
$F\from  \RRR^{2n+2}   \to \RRR^2$ satisfies the hypothesis of the Implicit functions theorem. We discuss the surjectivity of:
$$(dF)= \left( \begin{array}{c|c} \frac{\partial f}{\partial x} & \frac{\partial f}{\partial v} \\\hline \frac{\partial f'}{\partial x} & \frac{\partial f'}{\partial v} \end{array} \right)= \begin{pmatrix} J& 0 \\ m & J \end{pmatrix}$$
same structure as in the HW problem.
\begin{gather*}
\left(\begin{array}{ccc|ccc}
2x_1 & \ldots & 2x_{n+1} & & 0 & \\
2 & \ldots & 2 & 2x_1 & \ldots & 2x_{n+1}
\end{array} \right)\\
df_x\begin{pmatrix} x\\-1 \end{pmatrix} = \begin{pmatrix} 2x\cdot x \\0 \end{pmatrix}=\begin{pmatrix} 2  \\0 \end{pmatrix}\\
df_x\begin{pmatrix} 0\\x \end{pmatrix} = \begin{pmatrix} 0\\2 \end{pmatrix} 
\end{gather*}
In conclusion:
$$\tnsp S^n \subset \RRR^{2n+2}$$ 
Second Example: Let $f\from \RRR^3\to \RRR$
$$f(x,y,z)= z^2 + (\sqrt{x^2+y^2}-A)^2 -a^2$$
using polar coordinates: $z^2 + (r-A)^2 - a^2=0$\\
Using $r_x= \frac{x}{r}$ and $r_y= \frac{y}{r}$ 
\begin{align*}
(df)&=(f_x,f_y,f_z)\\
&=\left( \frac{2x}{r}(r-A), \frac{2y}{r}(r-A), 2z \right)
\end{align*}
$J$ has rank 1 in level set $T^2=f ^{-1} (0)$. Consider cases:
\begin{enumerate}
\item either $z\neq 0$ (0k)
\item $z=0$ and $f_x^2 + f_y^2 = (r-A)^2=4(r-A)^2=4a^2 \neq 0$ 
\end{enumerate}
    so $f_x,f_y$ not both 0.
\end{examples}
