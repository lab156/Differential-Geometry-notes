%20 de marzo
\subsection{Orientability in terms of vector fields of by vector fields}
A frame at $p\in M$: $\{e_1,\ldots,e_n \}$ basis for $\tnsp_pM$ determines an orientation at $p$.

If $\phi\from U_p \to \RRR^n$ then $d\phi_p(e_i)=f_i \in \RRR^n$. Now let $\{e_1,\ldots, e_n\}$ be $n$ smooth linear independent vectors fields on open set $U\subset M$.

Assume $M=\cup U_i$ covering and frames $F_i=\{e_1^i,\ldots, e_n^i\}$ for each $U_i$; assume $F_i,F_j$ consistent on $U_i\cap U_j\neq \empty$ then $M$ is oriented.

\begin{teorema}
Prove that any Lie group is orientable.
\end{teorema}

\begin{examples}
\hspace{1cm}
\begin{itemize}
\item $SO(3) \simeq \RRR P^3$ note that:
\begin{gather*}
B_\pi = \{x\in \RRR^3 \ |x| \leq \pi \}\\
\hat B_\pi = B_\pi/\{\pm \}
\end{gather*}
where $\{\pm\}$ is the equivalence relationship that identifies antipodal points. \\
$F\from B_\pi/\{\pm\} \to SO(3),\ F(x) = R_x(|x|)$.
\item $S^3=SU(2)$ is parallelizable.
\end{itemize}
\end{examples}

\subsection{The Lie Bracket}
\begin{ddef}
Given vector fields $X,Y\in X^\infty (M)$ that are smooth on $M$. Let an oprator on smooth functions $f\in C^\infty(M)$ called the \textbf{Lie Bracket} of $X$ and $Y$ defined as:
$$[X,Y](f) = X(Y(f)) - Y(X(f))$$
so it runs out that $[X,Y]: C^\infty(M) \to C^\infty(M)$. In fact it can be proven easily that $[X,Y]$ actis as a linear derivation a each $p\in M$; that is 
\begin{enumerate}[1)]
\item Linear 
\item Satisfies Liebnitz rule.
\item Evidently second order partial derivations cancel out.
\item And $[X,Y]=Z\in X^\infty(M)$ varies smoothly with $p\in M$ 
\end{enumerate}
\end{ddef}

\begin{teorema}
The Lie bracket fives the Infinite dimensional vector space $X^\infty(M)$ the structure of a Lie Algebra:\\
$\forall \; X,Y,Z \in X^\infty(M)\quad a,b\in \RRR$:
\begin{enumerate}[1)]
\item $[X,Y]= - [Y,X]$
\item $[aX+bY,Z] = a[X,Z] + b[Y,Z]$ (bilearity).
\item $[[X,Y],Z] + [[Y,Z],X] + [[Z,X],Y] = 0$ (Jacobi Identity).
\end{enumerate}
\end{teorema}

\begin{remarks}
The jacobi identity states $[[X,Y],Z] - [X,[Y,Z]] = [Y,[Z,X]]$, this says that since the right hand side is generally non-zero, then it is measuring the non-associativity of the Lie bracket.
\end{remarks}

\begin{ddef}
Let $\phi: M\to N$ be a smooth map  and $X\in X^\infty(M),\ A\in X^\infty(N)$.\\
Then $X$ is \textbf{$\phi$-related} to $A$ iff $d\phi(X) = A_\phi$ at each $p\in M$.  Another way to define this is that for each $g\in C^\infty(N)$ and $p\in M$,
$$A_{\phi(p)} (g) = d\phi (X_p) (g) = X_p(g\circ \phi) $$
suppress $p$, and we get $X(g\circ \phi) = A(g)\circ \phi$ and this is a very useful alternate definition of $\phi$-relatedness.
\end{ddef}

\begin{teorema}
Let $\phi\from M\to N$ be smooth. If $X$ is $\phi$-related to $A$. and $Y$ is $\phi$-related to $B$, then $Z=[X,Y]$ is $\phi$-related to $C= [A,B]$
\begin{proof}
For all $g\in C^\infty(\NN)$, $C(g)\circ \phi = [A,B](g)\circ \phi$
\begin{align*}
&= A(B(g))\circ\phi - B(A(g)) \circ \phi \\
&= X(B(g)\circ\phi) - Y(A(g)\circ\phi)\\
&= X(Y(g\circ\phi)) - Y(X(g\circ\phi))\\
&= [X,Y](g\circ\phi) = Z(g\circ\phi)
\end{align*}
\end{proof}
\end{teorema}

\begin{teorema}
If $\phi= (x_1,\ldots,x_n)$ is a chart on $U\subset M$ , then the coordinate field $\frac{\partial}{\partial x_i}  $ are $\phi$-related to standard fields $\frac{\partial}{\partial r_i} $ so any two coordinate vector fields commute.
$$\left[ \frac{\partial}{\partial x_i}, \frac{\partial}{\partial x_j}  \right]=0$$
\begin{proof}
Let $g\in C^\infty(\RRR^n)$, let $f=g\circ \phi$ then 
$$\frac{\partial}{\partial x_i} (g\circ\phi)=\frac{\partial f}{\partial x_i} = \frac{\partial f\circ \phi^{-1} }{\partial r_i} \circ \phi = \frac{\partial g}{\partial r_i} \circ \phi$$
Then $[\frac{\partial}{\partial x_i} , \frac{\partial}{\partial x_j}] = 0 $ because $[\frac{\partial}{\partial r_i} , \frac{\partial}{\partial r_j}] = 0$.
\end{proof}
\end{teorema}
