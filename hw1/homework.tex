%% DIFFERENTIAL GEOMETRY HOMEWORK 1
%% PROF JOEL LANGER
\begin{enumerate}
\item Describe $Gl_n(\RRR)$ as a lie group.\\
besides that the determinant can be used to proves that $Gl_n^+(\RRR)=\{A: \det(A)>0 \}$. A bit more interesting is that there seems to be no isomorphism possible between $Gl_n(\RRR)$ and $\RRR^n$ because there are non-singular matrices that do not conmute. 
\item Describe $P^3(\RRR)$ as a smooth, oriented manifold.\\
en este se usa una funcion de $\RRR^{n+1}$ en un abierto donde una componente no se haga cero.
\item Show that $S^2$ and $P^1(\CC)$ are naturally diffeomorphic.\\
If $\phi_1$ is the chart from $P^1(\CC)$ to $\RRR^2\ [z_1,z_2] \mapsto z_2/z_1$ ; and $\psi_1$ from $S^2$ to $\RRR^2\ (u,v)\mapsto \frac{u}{1-v}$. Then $\psi\circ f \circ \phi^ {-1} $ is:
\begin{equation*}
(s,t) \xrightarrow{\phi_1^ {-1} }[1, s+it] \xrightarrow{f} \left( \frac{(s,t)}{s^2+t^2+1},\frac{s^2+t^2-1}{s^2+t^2+1} \right)\xrightarrow{\psi_1} (s,t)
\end{equation*}  
\end{enumerate}
